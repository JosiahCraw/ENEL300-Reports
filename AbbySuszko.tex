\documentclass{article}
\usepackage[a4paper, total={6in, 9in}]{geometry}

\title{Dr Abby Suszko Guest Lecture Review}

\author{Josiah Craw\\35046080}

\begin{document}

\maketitle

In this lecture Abby went over an introduction to the interface between Electrical and Computer
Engineers and M\=aori. The first point covered by Dr Suszko was where might we find ourselves
interacting with M\=aori? These are as follows:

\begin{itemize}

    \item{Where Iwi and hap\=u are partners}

    \item{Where Iwi and hap\=u are shareholders in the project/company}

    \item{Where Iwi and hap\=u are stakeholders in the project}

    \item{Where Iwi, hap\=u, Wh\=anau, individuals and organisations are clients}

\end{itemize}

We as Engineers are now finding more Iwi and Hap\=u in these situations as there has been a move
from these people from being just land owners/leasing situations to more working directly with
engineers to complete their own projects.

The next point that Abby covered was the differences in introduction and the importance of the
order. With New Zealanders starting with who they are and what they do, where M\=aori starting with
where and who they come from before moving to how they are. The second main point of difference is
that concept of time where for New Zealanders the time something takes is important, but for M\=aori
if a meeting goes over it goes over, the time is inconsequential.

Abby then moved on to several examples where partnering directly with local Iwi and Hap\=u has been
beneficial to companies. The major example of this is Mercury where after begin receptive to input
and feedback they have managed to grow their company with the support of local Iwi and Hap\=u. With
Mercury owning and co-owning several successful power plants over mainly the North Island due to
this relationship. The next example of positive outcome in interactions for an electrical engineering
company was the Awarua Synergy example, this is where a local Iwi commissioned the installation of
three nine metre high wind turbines to power the marae and surrounding homes. And finally, for an
example of when a company refused to interact with local Iwi was when a company attempted to build
a wind farm near Napier. The company was originally granted resource consent however, after refusing
to speak to local Iwi about the location of the turbines the company was challenged in the
environment court and lost.

Finally, the key take aways for this lecture was to engage with M\=aori as this can be beneficial
to our company as the support can be invaluable. The next major takeaway was that when interacting
we must respect the correct way of doing things and that strong communication and relations and
compromise is key for everybody getting what they want.

\end{document}
