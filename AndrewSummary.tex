\documentclass{article}

\usepackage[a4paper, total={6in, 9in}]{geometry}
\usepackage{textcomp}

\title{Andrew McNamara Guest Lecture Summary}
\author{Josiah Craw\\35046080}

\begin{document}

\maketitle

Over Andrew's lecture he covered the design challenges of designing products for use in space and
for space vehicles. Andrew used an example from a product that he worked on, a Lithium Ion Battery
pack for a lunar imaging vehicle. The battery had a capacity of 100Ah and cost 2\$Million NZD, this
was the first project that Andrew was the team lead of.

Andrew first talked about environment that has to be considered when designing a product for space.
Some key points of this environment were: 

\begin{itemize}

    \item{No Air, this leads to:}
        \begin{enumerate}
            \item{No convective heat transfer}
            \item{Off gassing of VOC MoCs}
        \end{enumerate}
    \item{Electromagnetic Fields, this leads to:}
        \begin{enumerate}
            \item{Space plasmas, causing ionised gasses}
            \item{The ship is a loose conductor}
        \end{enumerate}
    \item{Radiation}
    \item{Thermal extremes, with the solar surface at 4000-6000\textdegree K}

\end{itemize}

Other issues in space is the lack of available power, and that launching costs money. Andrew next
covered the hypothetical of something breaking when it went to space and what events could cause the
system to stop working these are:

\begin{itemize}

    \item{Battery failure}

    \item{Temperature event}

    \item{Mechanical failure}

\end{itemize}

When then covered the potential cost to fix the system, an example of the high travel cost was given
in the Hubble telescope, where the BOM cost was \$95Million USD and the transport was \$110Million
USD. The lunar reconnaissance orbiter (LRO) that Andrew worked on would have cost \$400Million USD
to replace using the Falcon 9 rocket.

Next, Andrew discussed the difficulties in getting to space, where concerns are in rapid
depressurisation, shock loads which can get up to 3000G or more at the interface, and the vibration
load with can be between 10 and 100G at resonance. Next, Andrew discussed ground testing, he
discussed the difficulty of ground testing space equipment. The first part of ground testing is that
hardware must be tested in the exact same configuration as it will be in space and if any changes
occur all previously run test must be rerun on the updated system, this could be very expensive as
testing can take up to a year. The next challenge is manufacturing, where all the
components are assembled in clean rooms, following a meticulously designed process of assembly. This
is a difficult an repetitive task. The clean rooms have laminar air flow to ensure no particulate is
present in the components. Finally Andrew summed up the process, told us about the EESA and ECSS
standards for space design, these are strict guidelines and specifications covering every part of space and rocket
design. Andrew next talked about the system life cycle, where every system and subsystem on the LRO
has a defined and calculated life cycle.

Andrew finally told us about the points of analysis before the build of the LRO:

\begin{itemize}

    \item{Reliability}

    \item{Failure Modes}

    \item{Parts degrading}

    \item{Magnetic moment}
    
    \item{Mass budget}

    \item{Structural (Static)}

    \item{Structural (Dynamic)}
    
\end{itemize}

The takeaway components for this lecture was the carefully consider design and ensure all components
of the system are considered before use, also we should engineer to our environments.

\end{document}
