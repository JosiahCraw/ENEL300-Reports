\documentclass{article}

\usepackage[a4paper, total={6in,9in}]{geometry}

\title{Donald McKellar Guest Lecture Summary}

\author{Josiah Craw\\35046080}

\begin{document}

\maketitle

Donald works for Allied Telesis, a networking company. He works primarily in the testing space where
he works with software. Donald first spoke to us about the way that Allied Telesis does development
planning. Allied Telesis has four main groups interacting in the development process, marketing,
customer service, software management, and software development. The marketing team give the market
requirements to the software management unit, this unit also takes input from the customer service
unit, in the form of customer issues. The software management then passes sets of jobs to be
completed to the developments teams. These could be high priority issues, items that had been
backlogged and new projects. Donald next discussed the work flow of the company out lining their
structure and use of Git the version control system. The company uses multiple branches in Git with
a main code stream branch, release branch and development branch. The first branch he discussed was
the release branch, which is the branch where software versions are released and bug fixed until
their end of life. The next branch discussed was the development and test branch, which is split off
the main branch and is regularly rebased the keep up with changes from the main branch, once features
are implemented and tested in this branch they are merged back into the main code stream. The main
code stream is largely kept unchanged in the days before a new release as the merged main code
stream is throughly tested. Donald next discussed design and the team structure used at Allied
Telesis. Allied Telesis has software development teams consisting of seven software developers and
one or two test engineers. An external test engineer checks the test planning and implementation and
assists in the testing phase. These teams when assigned to a team, are given a feature to implement
all the engineers then brain storm and contribute idea to the idea. A set of specifications is then
developed for the implementation. Donald then discussed the methods used by Allied Telesis. The
preferred method of development is agile, with the scrum method used in teams. Projects are broken
down multiple times until the resulting tasks can be completed by an individual team member. Donald
then told us that at Allied Telesis daily meetings are held where progress is discussed where each
team member says what they did the previous day and what they plan to do that day. Next where were
told the all code is peer reviewed before the code is added to the project stream to reduce the
number of mistakes made. The next topic discussed was testing. At this stage all new features are
tested as well as testing for issues that may have arisen to other features. At this stage the tests
that are used are also maintained. Challenges is this area are issues with code degradation, the
many platforms that are being developed on, the huge amount of features as no feature is ever
removed, issues with the ASICs made for the devices and finally the required reduction in testing
time. Overall, the take aways for this lecture are:

\begin{itemize}

    \item{Ensuring teams are well structured}

    \item{Ensure tests are well designed and maintained}

    \item{Make sure all tasks are broken down well enough to be accomplished}
    
    \item{Make sure all work is checked by someone else preferably external to the project}

\end{itemize}

\end{document}
