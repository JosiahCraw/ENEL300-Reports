\documentclass{article}

\usepackage[a4paper, total={6in, 9in}]{geometry}

\title{Mofassir Ul Haque Guest Lecture Summary}

\author{Josiah Craw\\35046080}

\begin{document}

\maketitle{}

In this guest lecture we covered secure software, vulnerabilities, exploitation, and tools that 
are used within the industry. First, Mofassir discussed what a vulnerability is, they are usually
a design flaw, an implementation bug or an improper implementation. Some examples of these are,
the design of the internet, as a design flaw as it was designed before hacking of internet
vulnerabilities was considered. This is an issue now as the internet has expanded and
vulnerabilities are a key design consideration. The next example covered an implementation bug,
this is when a programmer has a code based flaw that could be exploited. The final example is an
improper configuration, this is when the software is designed well and there are few to no
exploitable flaws however, the end user has set the software up poorly like leaving the password
as the default.\\

Mofassir then covered exploitations, these are when a person of party find a vulnerability and
exploit it some examples are given.

\begin{itemize}
    
    \item{Loss of sensitive information}

        \begin{itemize}
            
            \item{Capital One Bank lost more than 100 million records}

            \item{The European GDPR allows a fine for a loss equal to either 10 million Euro or
                4\% of a companies annual turnover}
                \begin{itemize}
                    
                    \item{British Airways lost 500,000 records and received a 200 million Euro fine}
                        
                    \item{Marriot Hotel lost 3 million records and received a 110 million euro
                        fine}

                \end{itemize}
        
        \end{itemize}
        
        \item{DDoS GitHub had 1.35 Terabits per second of traffic through a DDoS attack in the 1st
            quarter of 2019}

        \item{Firewalls, Anti-Virus, IDS, IPS are not particularly effective}

\end{itemize}

The next component of the lecture was on the tools used by hackers and professionals as well as
governments to access different software. Mofassir told us these can be broken down into three
categories:

\begin{itemize}
    
    \item{Free tools}

    \item{Paid tools}

    \item{Custom tools, these are mostly used by governments}

\end{itemize}

First we discussed free tools, these tools usually require some simple computer experience. These
include Nmap, Nexpose, Nessus, Metasploit, Armitage, Cain \& Abel, John the Ripper, Hydra, and
Aircrack. These tools are effective and easy to access. The next set of tools we discussed were
paid tools, these can be bought by individuals and are usually purchase for only a few thousand
dollars. Usually these tools are purchased on the dark web. The final tier of tools are custom
tools used by governments. These tools include:

\newpage

\begin{itemize}

    \item{NSA tools}

        \begin{itemize}
            
            \item{UNITEDRAKE - Gain total control of a computer}

            \item{Captivated audience - Used for recording using a user microphone}

            \item{Gum Fish - Uses users webcam to record/take pictures}

            \item{Foggy - Used to get internet history and passwords}

            \item{Grok - A keylogger}

            \item{Salvage Rabbit - Used to access external media on a PC}

        \end{itemize}
    
    \item{Pegasus - Designed by NSO}

        \begin{itemize}
            
            \item{Used by govenments}

            \item{Works on all operating systems}

            \item{Is able to access data such as messages, location and accessing the webcam and
                microphone}

        \end{itemize}

\end{itemize}

The next topic we discussed was the people that exploit vulnerabilities these are: hackers, state
actors, and activists. Hackers try and get access to secure systems, usually for profit. State
actors are people working on behalf of a government and either attempt to secure local systems
and exploit foreign systems for their government. Finally, activists these people attempt to gain
access to systems in an attempt to further an agenda, eg. A more open access internet or better
privacy. Examples of these activists are, Aron Swartz, Edward Snowden and Anonymous.\\

The next topic of discussion was finding and reporting vulnerabilities. The method described by 
Mofassir was through bounty programs where companies offer money in exchange for discovering
software issues in their platforms. The amount of money is dependent upon the severity of the flaw
and the ease at which an intruder could gain access. The next stage is to report the
vulnerability to the relevant CERT organization. These organizations inform the relevant company
about the flaw and give a time period before they release the flaw publicly. This is to ensure the
company responds to the issue. When made public the flaw is sent on to various mailing lists, this
allows users to update the software to the version where the issue is fixed.\\

The final discussion topic was implementation vulnerabilities these include, improper memory
allocation, improper input validation, and improper protocol implementation. Improper memory
allocation allows hacker to write data over the buffer end, potentially giving access beyond what
is intended. Improper input validation allows hackers to inject unwanted scripts/requests into
code. The final vulnerability allows the hacker to exploit a protocol that was implemented poorly
such as using a TCP sync attack.\\

The key take away points for this lecture was to ensure your code is secure and that we should
always update to the latest security patch.
\end{document}
