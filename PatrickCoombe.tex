\documentclass{article}

\usepackage[a4paper, total={6in, 9in}]{geometry}
\usepackage{mathastext}

\title{Patrick Coombe Guest Lecture Summary\\Manap$\bar{o}$uri Local Services Upgrade}

\author{Josiah Craw\\35046080}

\begin{document}

\maketitle

This guest lecture was presented by Patrick Coombe, a member of Mitton ElectroNet's safety team.
Mitton is an electrical engineering consultancy firm based in Christchurch with approximately 80
staff. This lecture covered the ongoing upgrade to the Manap$\bar{o}$uri Hydroelectric power
station. This power station was completed in 1971 it is the largest hydroelectric power station in
New Zealand. The plant was originally built to power the Tiwai Point aluminum smelter. The smelter
is located 160km and is one of the largest energy consumers in New Zealand. If the power stop at
the smelter for more than two hours the aluminium will harden in the pots and need to be blasted
out, this would cause the aluminium smelter to be off for approximately six months. Under contract
Meridian must supply 600MW to the smelter and if this does not happen and Meridian did not make
their best effort to ensure this Meridian must pay a large sum to the smelter.

Patrick then discussed what would happen if the power station lost power. After two hours the 
underground area would turn off, and after twelve hours above ground regulation would fail.
Patrick then talked about the design difficulties faced during the project to date. The first of
which was the control system that was currently used at the station. This system was very old and
no one working on the project knew how it worked complete so all upgrade works had to be careful
as if the existing system was changed so that the controller thought the system failed it could
cause power outages. This was not an option also the station was not able to be switched off
during upgrade due to the contract with the aluminium smelter. Over the course of the project the
staged approach had to be changed around six times and the six month upgrade project is now closer
to twelve months, as the engineers now have more knowledge to how the existing system works and
know about more system issues. The next issue was the diesel generators nearing their end of life
this as well as poor control over these generators meant that these had to be replaced. The next
major issue was that the 220/11kV local service transformers were nearing their end of life.

Patrick then discussed the potential issues that could arise while upgrading, the first of which
was if both below ground 400V supplies were lost then multiple generators would stop production.
Or if the transformer oil pump failed or lost power this could cause generators to start shutting
down. This systems failure could have huge issues for the New Zealand power grid as well as the
Tiwai smelter.

Patrick then discussed the upgrades that were required as a part of the upgrade. The first of
which was replacement of the end of life equipment. The next was a more reliable control system 
had to be installed. Finally, a more reliable diesel backup system had to be installed. The
existing oil transformer was replaced with a gas insulated one, this eliminated the fire and
containment risks associated with oil cooled transformers. The next upgrade was the control system
which was replaced with an entirely new system this was designed with simpler code to help with
maintainability. The system was also upgraded to have more reliable above and below ground power.

The major takeaways for this lecture were primarily on risk management. The mains points were that
risks are present in all projects and must be mitigated to the best of an engineers abilities.
Patrick also talked about the acceptance for risk may change throughout the project may change.
Patrick told us about an example from this project were management changed their acceptance for
risk from N security to N-1 security, this meant the engineers in this project had to redesign the
solution. The rest of the take aways were:

\begin{itemize}
    
    \item{Keep designs simple}

    \item{Complex solutions may not be the best solution}

    \item{Communication is very important}

    \item{Regularly discuss things that are learnt over a project}

    \item{Engineering in remote locations can be difficult}

\end{itemize}

\end{document}
