\documentclass{article}

\usepackage[a4paper, total={6in, 9in}]{geometry}

\title{Professor Bones Guest Lecture Summary}
\author{Jos Craw}

\begin{document}

\maketitle{}
For the lecture with Professor Bones the design of everyday things was discussed. The main points
discussed were the following:

\begin{itemize}
    \item{Affordances}
    \item{Signifiers}
    \item{Mappings}
    \item{Feedback}
    \item{Conceptual Models}
    \item{Constraints}
    \item{Conventions}
\end{itemize}

These principals make up the core components of design. The first is affordances, this is visually
what the design allows for i.e. a chair affords sitting. This shows that a design should allow a
potential user to look at it and know what is is used for and preferable how to use it.\\

The second is siginfiers, these include obvious symbolism such as arrow icons on an application
signifying that there are more pages. Another example of siginfiers is a simple sign such as a
traffic sign show what lane is for what turn off. Signifiers are important especially for complex
systems as it allows the user to quickly understand a design.

The next point discussed was mappings, mapping is where the operation of something is
obvious though it's shape or location. Another provided example was the London Underground map as
the map is not to scale but simply is a good representative of the direction and flow of the
system.\\

The next item discussed was feedback this is when an operation returns a response to indicate that
the operation was preformed correctly. The given example for this is when a light switch is
pressed and the feedback is the light coming on. Feedback is important as it shows users that an
action was completed successfully this is important if otherwise there would be no indication of
the design operating correctly.\\

Next, conceptual models were discussed. The main example for this was the location of light
switches on doors. The usual location is on the latch side beside the door, this allows the door
to be opened and the light switch to be operated in one motion. An example issue introduced to
counter this assumption was the case of cat and kitten doors and which side should the switch go
in this case. After discussion it was determined that the light switch should be on the latch side
of the larger door as this would be by far the most used door.\\

The final discussion was about conventions. The lack of conventions for many things was discussed
such as hot and cold water taps and which side each temperature is on. Also discussed was the lack
of conventions for a T.V. remote and how this causes user confusion and frustration as the design
is difficult to use.

\end{document}
