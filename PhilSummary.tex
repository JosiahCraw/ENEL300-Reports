\documentclass{article}

\usepackage[a4paper, total={6in, 8in}]{geometry}

\title{Professor Bones Guest Lecture Summary}
\author{Jos Craw}

\begin{document}

\maketitle{}
For the lecture with Professor Bones the design of everyday things was discussed. The main points
discussed were the following:

\begin{itemize}
    \item{Affordances}
    \item{Signifiers}
    \item{Mappings}
    \item{Feedback}
    \item{Conceptual Models}
    \item{Constraints}
    \item{Conventions}
\end{itemize}

These principals make up the core components of design. The first is affordances, this is visually
what the design allows for i.e. a chair affords sitting.\\

The second is siginfiers, these include obvious symbolism such as arrow icons on an application
signifying that there are more pages. Another example of siginfiers is a simple sign such as a
traffic sign show what lane is for what turn off.\\

The next point discussed was mappings, mapping is where the operation of something is
obvious though it's shape or location. Another provided example was the London Underground map as
the map is not to scale but simply is a good representative of the direction and flow of the
system.\\

The next item discussed was feedback this is when an operation returns a response to indicate that
the operation was preformed correctly. The given example for this is when a light switch is
pressed and the feedback is the light coming on.\\

Next, conceptual models were discussed. The main 

\end{document}
