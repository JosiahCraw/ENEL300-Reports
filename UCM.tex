\documentclass{article}

\usepackage[a4paper, total={6in, 9in}]{geometry}

\title{UCM Guest Lecture Summary}
\author{Jos Craw\\ 35046080}

\begin{document}

\maketitle{}

The UCM team compete every year in the Formula SAE competition, this competition has each
competing student team build and race a formula style car. At the beginning of UC's involvement
the team competed in the petrol engine category where, on the final year that UC competed they
placed third. The following year the UCM team switched to racing in the EV category. This switch
was a result of the belief from the team that electric is the future and that the team should be
attempting to learn the skills required to design and build an EV. The Formula SAE competition is
judged in the five following categories.

\begin{itemize}
    \item{Acceleration}
    \item{Skid Pad}
    \item{One Lap around the track}
    \item{Endurance}
    \item{Efficiency}
\end{itemize}

When initially designing the car several design choices had to be made. The first of which was the
type of drive train used, either four wheel drive or two wheel drive. The UCM team chose four
wheel drive due to its better performance, grip and it's ability to incorporate regenerative
breaking. The second  design choice was the which cells to use in the accumulator of the vehicle.
The UCM team did a simulation to determine the best cell choice and the result was to use lithium
polymer pouches. However, due to the difficulty in ensuring genuine batteries at the time this was
not an available option. The next option was to use Samsung lithium ion 18650 cells spot welded
together. This option also did not work as the spot welding was not reliable. The option that was
used was ultrasonically coupled lithium ion 18650.\\

The next category of the car that was discussed was the HV front end. The first component of which
is the inverter, due to the workings of inverter it has high input capacitance. This high input
capacitance necessitates the next component, the pre charge circuit. The pre charge circuit uses a
resistor to charge the inverter before opening the circuit across the inverter. The next topic was
the battery improvement from 2018 to 2019, the UCM team will be changing the current 8 parallel
setup to 7 parallel. This upgrade was as the 8 parallel was too hot at nominal voltage the upgrade
also saved weight and increased voltage.\\

The next topic discussed was the critical PCB's which are listed below.

\begin{enumerate}
    \item{Isolation board}
    \item{Fault latch circuit}
    \begin{itemize}
        \item{Analog}
        \item{If main the main control system fails or a fault occurs this circuit will shut the system down}
    \end{itemize}
    \item{Can bus interface}
    \begin{itemize}
        \item{Eliminates/reduces noise for analog communication}
    \end{itemize}
    \item{Discharge circuit}
    \begin{itemize}
        \item{Contains high speed communication}
        \item{Discharges from the high voltage circuit}
    \end{itemize}
    \item{Vehicle control unit (VCU)}
    \begin{itemize}
        \item{Contains traction control}
        \item{Contains torque vectoring control}
        \item{Contains cooling control}
    \end{itemize}
\end{enumerate}

The final topic covered in the technical section was the future of the UCM design. There are three
major improvements that the UCM team is attempting to integrate.

\begin{itemize}
    \item{Use a self developed silicon carbide MOSFET instead of the current IGBT for improved
        efficiency}
    \item{Use lithium polymer pouch as the technology is now more reliable}
    \item{And to develop driver less technology as this will be a requirement by 2020}
\end{itemize}

The final section of the lecture was covering the way contributing to UCM can help students after
graduation. This section covered how UCM shows potential employers that a student has practical
experience and team working skills that are highly valued in the professional space. 

\end{document}

